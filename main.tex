\documentclass[11pt]{article}
\usepackage[margin=1in]{geometry}
\usepackage{amsmath}
\usepackage{amssymb}
\usepackage{amsthm}
\usepackage{algorithm}
\usepackage{algorithmic}
\usepackage{graphicx}
\usepackage{hyperref}
\usepackage{optidef}
\usepackage{enumitem}
\usepackage{tikz}
\usetikzlibrary{arrows.meta,positioning,shapes}

% Theorem environments
\newtheorem{theorem}{Theorem}
\newtheorem{lemma}[theorem]{Lemma}
\newtheorem{proposition}[theorem]{Proposition}
\newtheorem{corollary}[theorem]{Corollary}
\newtheorem{definition}{Definition}
\newtheorem{remark}{Remark}

% Custom commands
\newcommand{\R}{\mathbb{R}}
\newcommand{\Z}{\mathbb{Z}}
\newcommand{\N}{\mathbb{N}}
\DeclareMathOperator*{\argmax}{arg\,max}
\DeclareMathOperator*{\argmin}{arg\,min}
\def\definitionautorefname{Definition}
\title{Geometric Programming to Petri Net Pipeline:\\
A Novel Approach for Probabilistic Planning via Structural Analysis}

\author{
Anonymous Authors\\
\textit{Department of Computer Science}\\
\textit{Institution Name}
}

\date{\today}

\begin{document}

\maketitle

\begin{abstract}
  We present a class of stochastic problems with mixed fully and
  non-observable state variables, and a novel pipeline for
  transforming these probabilistic planning problems into Petri Net
  representations via Geometric Programming (GP) condensation. The key
  innovation is using GP condensation not for optimization, but to
  generate weighted sum structures suitable for Petri Net analysis. By
  treating binary decision variables as continuous during GP
  formulation, applying arithmetic-geometric mean inequality for
  condensation, and subsequently enforcing binary constraints through
  Big-M encoding in log-space, we enable sophisticated structural
  analysis while maintaining discrete semantics. Our approach
  discovers mutex relationships, invariants, and reachability bounds
  that significantly accelerate SMT solving. We demonstrate the
  effectiveness on the probabilistic Park Rangers' Problem, showing
  that preprocessing via Petri Net structural analysis can reduce
  solving time by orders of magnitude compared to direct SMT encoding.
\end{abstract}

\section{Introduction}

Various methods for solving stochastic sequential decision problems
exist today in the literature. The Markov Decision Process (MDP) and
its partially observable variant the Partially Observable Markov Decision Process
(POMDP) are classical methods for modeling these problems with
numerous solution algorithms available, both exact and
approximate. Probabilistic planning problems with discrete decisions
and continuous uncertainty, though they can be modeled as POMDPs,
present significant computational challenges, especially if optimal
solutions are desired.

We present a class of problems which have some non-observable
stochastic components which prevent the problems from being modeled as
MDPs, and which may be modeled as POMDPs but for which traditional
POMDP solution methods are unnecessarily complex.

We then present an algorithm for solving this class of problems which
leverages the mathematical foundations of geometric programming to
formulate the problem as a mixed-integer linear program or MILP.

\subsection{Contributions}

Our main contributions are:

\begin{enumerate}
\item A definition of an intermediate class of stochastic problem, the
  Mixed Observable-Non-Observable Markov Decision Process (MONOMDP).
\item A complete pipeline for transforming MONOMDPs with transition
  functions composed of posynomial constraints with some binary
  decision variables arising in probabilistic planning into analyzable
  Petri Net structures.
\item A Big-M encoding scheme for handling binary variables in
  log-space while preserving discrete semantics.
\item Demonstration that GP condensation can serve as a preprocessing
  tool for structure discovery rather than optimization.
\end{enumerate}

\section{Background}

\subsection{Markov Decision Process}

A \emph{Markov Decision Process} or MDP is a framework for solving
stochastic problems with fully observable states and stochastic action
outcomes, with additive rewards.

An MDP is composed of an initial state $s_0$, a set of actions $A$, a
probabilistic transition model $P(s' | s, a)$ and a reward function,
$R(s,a,s')$. A solution to an MDP is a policy $\pi(s)$ which, given
any state, gives an action that should maximize the expected utility
of the next state. All states are \emph{fully observable} -- that is,
although an agent may not know the outcome of an action before it is
taken, once an action is taken, the outcome is immediately
known \cite[Chapter 17]{AIRusselAndNorvig}.

\subsection{Partially Observable MDPs}

A \emph{Partially Observable Markov Decision Process} (POMDP)
addresses some limitations of MDPs when applied to the real world --
namely, that an MDP must know exactly what state it is in.

A POMDP instead deals with \emph{belief states}, where the belief
state $b$, a probability distribution over all the discrete states,
takes the place of the known state $s$ in an MDP, and a POMDP
additionally has a sensor model $P(e|s)$ which is used to update the
belief state after taking actions. This new problem space is
continuous and of a much higher dimension than the corresponding MDP
would be if the state were fully observable, but we similarly want a
policy $\pi(b)$ which maps a current belief state to an
action \cite[Chapter 17]{AIRusselAndNorvig}.

\section{Problem Formulation}

In this paper, we are interested in addressing a set of problems which
are Markovian, and have a mix of fully observable state components and
other state components which can be estimated only using information
given \emph{a-priori} -- that is, state components which we cannot
update the estimation for using percepts during the solution of the
problem, but which we can estimate the value of based on some
information known about the problem \emph{a-priori}.

% More abstract -- ??
It would be possible to model these problems as POMDPs at the cost of
unnecessary computational expense -- we present a method that allows
for getting optimal solutions for this class of problems given a fixed
planning horizon.

\subsection{MONOMDPs}
\begin{definition}{Mixed Observable-Non-Observable Markov Decision Process (MONOMDP) }\label{def:monomdp}

  We describe a \emph{Mixed Observable-Non-Observable Markov Decision
  Process} (MONOMDP) $M = (S, A, p, s_0, g)$ where:

  % 0. NP or P-Space?

  % 1. Distinguish between states/actions (add an action to the tf)
  %

  % 2. The actual value of whether or not we were seen is boolean, we
  % have a probability estimate. So there are variables we can control
  % and ones we can't which we have some estimation for.

  % 3. Separate state and action variables.

  % 4. Uncertainty is part of start state, (??)

  % 5 separate states/actions, binary state + set of actions,
  % uncertainty is start state, it is a set of probabilities.

  % What is the desired output? A sequence of actions.
  \begin{itemize}
  \item $S$ is the state space A state $s \in S$ is composed of $m$
    boolean state elements $B(s) = (b_{0}...b_{m-1})$ which are fully
    observable, and $n$ boolean state elements which are not
    observable but for which we have some probabilistic model $R(s) =
    (r_{0}...r_{n-1})$. We then define the state $s \in S$ as $s =
    (b_0,...b_{m-1},r_0,...r_{n-1})$. For compactness of notation we
    sometimes refer to the components of $s$ as $s_0...s_{m+n-1}$,
    indicating by the index value when necessary if a state variable
    is in $R(s)$ or $B(s)$.

  \item A set of actions $A$.

  \item The set $S'$ of state estimations where some $s' \in S'$ is
    composed of the regular boolean variables $B(s)$ and a probability
    estimate for each $R(s)$: $s' =
    (b_0,...,b_{m-1},P(r_0=\text{true}),...,P(r_{n-1}=\text{true}))$.

    % ^ Roll this down into the transition function. "If you want to
    % define a variable for your state knowledge that is also good".


    % Can/should we model it as only probabilities? Does our approach
    % absolutely depend on binary discrete variables in the problem
    % definition?

  \item A transition model, $p: S' \times A \rightarrow S'$, which
    takes a state and action and returns a tuple $(b_{0}', b_{1}',
    .... b_{m-1}',
    P(r_0'=\text{true}),...P(r_{n-1}'=\text{true}))$. This models both
    some components which have a predetermined outcome, and others
    which have a probabilistic outcome. The probabilistic $r_i$
    components cannot be observed during planning or plan execution.

  \item Some initial state $s'_0$.

  \item $g: S \rightarrow \mathbb{R}$, a utility function.
  \end{itemize}

  A solution to a MONOMDP problem is a sequence of actions
  $(a_0,a_1...a_N)$ which makes a path from the initial state $s'_0$ to
  the state of maximum estimated utility, $s'_N$.

\end{definition}

% Draw links with POMDPs and MDPs and discrete search.
% Is this also an HMM? If so draw the analogy here.

Unlike MDPs or POMDPs, a solution to a MONOMDP is not a policy $\pi$
mapping a state or belief state to a next action. This is because no
additional information can be observed during an attempt at solving
the problem, unlike MDPs and POMDPs, because actions only have either
fully deterministic effects or effects which cannot be observed.

Although this difference seems fundamental, MONOMDPs can be modeled as
POMDPs with a trivial sensor model. Since the $r_i$ variables are
non-observable and there is no sensor model or observations with which
to inform our state estimation, using POMDPs is unnecessarily complex.


\subsection{MONOMDPs with Posynomial Constraints}

In this paper, we present an algorithm for solving a subclass of
MONOMDPs where the transition model $p$ is composed of posynomial
constraints, and the objective $g$ is also a posynomial, and there is
some fixed time horizon. First we review the definition of
posynomials, then present the restricted set of MONOMDPs we propose to
solve using this definition.


\subsubsection{Posynomial Functions}

A posynomial is a function of the form:

\begin{equation}\label{eq:posys}
f(x) = \sum_{k=1}^{K} c_k x_1^{a_{1k}} x_2^{a_{2k}} \cdots x_n^{a_{nk}}
\end{equation}

where $c_k > 0$ for all $k$, and $a_{ik} \in \mathbb{R}$, and $x_i$
are nonnegative reals. Each term in the sum is called a monomial.


\begin{definition}{MONOMDP with posynomial constraints}\label{def:monomdp-pcstr}
  A MONOMDP with posynomial constraints places the following
  additional requirements on the problem structure:

  \begin{itemize}
  \item $p: S' \times A \rightarrow S'$ is composed of a set of
    posynomial constraints that govern the state transition. The
    probabilistic components of $s'$ or $R(s')$ relate the state
    component $r_i$ at step $k+1$ to the state at step $k$ as follows:

        \begin{align}
          P(r_{i,(k+1)} = \text{true}) &= \sum_{j} c^{\top}_j \prod_{l} s'_{l,k}^{a_{j,l}} \\
          P(r_{i,(k+1)} = \text{false}) &= \sum_{j} c^{\bot}_j \prod_{l} s'_{l,k}^{b_{j,l}}
        \end{align}


        While the discrete components relate the value of component
        $b_i$ at step $k$ to the value at step $(k+1)$ as follows:

        \begin{align}
          b_{i,(k+1)} = \sum_{j} c^{\top}_j \prod_{l} b_{l,k}^{d_{j,l}}
        \end{align}
      \item $g : S' \rightarrow \mathbb{R}$ must also be a posynomial
        over the state variables.
  \end{itemize}
\end{definition}

\section{Approach}

To solve a MONOMDP with posynomial constraints, we unroll the
transition function over $M$ steps and transform the resulting
equations into a geometric program. Per \autoref{def:monomdp-pcstr}, the
transition function already has the general form of a posynomial. In
our case, we are interested not in using off-the-shelf geometric
programming solvers to help solve our problem, but instead in the
power of the log transformation critical to geometric programs to
enable further analysis.

\subsection{Geometric Programming}

A geometric program (GP) is a type of mathematical optimization
problem with a special structure that allows for efficient solution
methods. A GP in standard form consists of minimizing a posynomial
objective function subject to posynomial inequality constraints and
monomial equality constraints.

\subsubsection{Mathematical Formulation}

The standard form of a geometric program is:

\begin{align}\label{eq:gp-def}
\text{minimize} \quad & f_0(x) \\
\text{subject to} \quad & f_i(x) \leq 1, \quad i = 1, 2, \ldots, m \\
& g_j(x) = 1, \quad j = 1, 2, \ldots, p \\
& x_k > 0, \quad k = 1, 2, \ldots, n
\end{align}

where $f_0(x), f_1(x), \ldots, f_m(x)$ are posynomial functions and
$g_1(x), \ldots, g_p(x)$ are monomial functions.

Geometric programs possess several attractive properties: they are
convex after a logarithmic change of variables, have polynomial-time
solution algorithms, and arise naturally in many engineering
applications including circuit design, control system design, and
resource allocation problems. Our constraints in \autoref{def:monomdp-pcstr}
are all posynomial.

\subsection{Condensation via Arithmetic-Geometric Mean}

The arithmetic-geometric mean inequality enables condensation of
posynomials into monomials.

\begin{theorem}[AM-GM Inequality]

  For positive weights $\epsilon_i$ with $\sum_i \epsilon_i = 1$ and
  positive terms $w_i$:

\begin{equation}
    \sum_{i} w_i \geq \prod_{i} \left(\frac{w_i}{\epsilon_i}\right)^{\epsilon_i}
\end{equation}

with equality when $w_i/\epsilon_i$ is constant for all $i$.
\end{theorem}

Given a posynomial constraint $f(x) \leq 1$, we create the condensed
monomial at point $\hat{x}$:

\begin{equation}
    \tilde{f}(x, \hat{x}) = \prod_{i} \left(\frac{w_i(x)}{\epsilon_i(\hat{x})}\right)^{\epsilon_i(\hat{x})}
\end{equation}

where $\epsilon_i(\hat{x}) = w_i(\hat{x})/f(\hat{x})$. This gives us a
function equal to the original function $f$ at the condensation point, and
less than $f$ around it.

\begin{proposition}
The condensed constraint $\tilde{g}(x, \hat{x}) \leq 1$ is conservative:

\begin{enumerate}
    \item $\tilde{f}(x, \hat{x}) \leq f(x)$ for all $x > 0$
    \item $\tilde{f}(\hat{x}, \hat{x}) = f(\hat{x})$
\end{enumerate}
\end{proposition}

\subsection{Log-Space Transformation}

Applying the logarithmic transformation $z_j = \ln(x_j)$ linearizes
the condensed monomials:

\begin{equation}
    \ln(\tilde{g}_k) = \sum_j \phi_{jk}(\hat{x}) \cdot z_j \leq 0
\end{equation}

\section{Transformation}

In order to leverage the AM-GM inequality to transform our problem in
\autoref{def:monomdp-pcstr} into a linear program, we must first overcome
two obstacles:

\begin{itemize}
\item We have binary variables which must take on the values 0,1, and
  geometric programs are by definition continuous. This we get around
  simply by not attempting to solve a traditional geometric program --
  instead, we are interested in using the mathematical tools provided
  by geometric programming to provide us with a convex (in this case
  linear) transition system for analysis (more below).

\item We have \emph{equality} constraints in our transition function
  $p(s')$, and posynomial constraints $f_i(x)$ as defined in
  \autoref{eq:gp-def} must take the form $f_i(x) \leq 1$ so that we
  may linearize them.
\end{itemize}

\subsection{Equality Constraints}

Our transition function defined in \autoref{def:monomdp} defines the
relationship between current and subsequent variables over a time
horizon of $M$ steps in this way:

\begin{align*}
  P(r_{i,(k+1)} = \text{true}) &= \sum_{j} c^{\top}_j \prod_{l} s'_{l,k}^{a_{j,l}}\quad  \forall 0 \leq k \leq M \\
  P(r_{i,(k+1)} = \text{false}) &= \sum_{j} c^{\bot}_j \prod_{l} s'_{l,k}^{b_{j,l}}\quad  \forall 0 \leq k \leq M \\
\end{align*}

But a posynomial constraint $f_i(x)$ in a GP, because of our use of the
arithmetic-geometric inequality mean, must take the form:
 
\begin{equation}
f_i(x) \leq 1, \quad i = 1, 2, \ldots, m
\end{equation}

In order to preserve the equality constraint, we transform this
equality into a bounded range for both values of $P(r_{i,(k+1)})$. We
write out the process for just the $P(r_i = \text{true})$ variable,
other variables follow the same process:

\begin{align*}
  \alpha P(r_{i,(k+1)} = \text{true}) &> \sum_{j} c^{\top}_j \prod_{l} s'_{l,k}^{a_{j,l}},\quad \alpha > 1 \\
\end{align*}

How tightly bounded the variables are depends on the selection of the
$\alpha, \beta$ values. Then the first equation easily takes the
desired form:

\begin{align}
  1 &> \frac{(\sum_{j} c^{\top}_j \prod_{l} s'_{l,k}^{a_{j,l}})}{P(r_{i,(k+1)}) \alpha} \label{eq:posy-bound-high}\\
\end{align}



\autoref{eq:posy-bound-high}, \autoref{eq:posy-bound-low} are
posynomials since dividing posynomials by a monomial results in a
posynomial. The resulting set of unrolled constraints can be
subsequently condensed around some initial, known-feasible point
$s_f$, (most likely the initial state $s'_0$ will be fully known and
can be used), and the resulting monomial can be log transformed. The
general result of a log transform on a monomial, which results in a
linear equation, is as follows:


\begin{align} \label{eq:log-transform}
  \ln(f(x)) &= \ln(c_k x_1^{a_{1k}} x_2^{a_{2k}} \cdots x_n^{a_{nk}}) \nonumber \\
  \ln(f(x)) &= \ln(c_k) + a_{1k}\ln(x_1) + a_{2k}\ln(x_2) + \cdots a_{nk}\ln(x_n) \nonumber \\
  \tilde{f}(x) &= \ln(c_k) + a_{1k}\tilde{x}_1 + a_{2k}\tilde{x}_2 + \cdots a_{nk}\tilde{x}_n
\end{align}

As in \autoref{eq:log-transform}, for some state component $s'_i$,
we'll refer to the log-transformed variable that appears in the
log-transformed linear equations as $\tilde{s'}_i$. When convenient
and unambiguous, we'll also refer to the discrete, binary components
of $s_i$ that have been log-transformed as $\tilde{b}_i$ and the
probabilistic ones as $\tilde{r}_i$.

\subsection{Binary Variable Encoding in Log-Space}

Our goal is to transform our problem into a linear program by log
transforming a condensation of each posynomial constraint. However, in
log-space, binary variables present a fundamental challenge:

\begin{align}
    b = 1 &\Rightarrow \ln(b) = 0 \\
    b = 0 &\Rightarrow \ln(b) = -\infty
\end{align}

That is, variables equal to 0 or $\text{false}$ cannot be directly solved for
in log-space. Although the condensed program posed as a normal linear
program can be solved as a linear program, there is no guarantee that
the values of our binary variables are actually binary, and any scheme
to enforce them to be binary must also overcome the fact that a 0
(i.e. $\textbf{false}$) value for a binary decision variable in our
original problem cannot be solved for directly in the transformed
problem.

\subsection{Big-M Encoding}

We resolve this through a kind of Big-M encoding scheme:

\begin{definition}[Big-M Binary Encoding]
  For each binary variable $b_{l,k} \in \{0,1\}$, we define \emph{in
  log-space}:
  \begin{align}
    \tilde{b}_{l,k} &\in \{-M, 0\} \\
    \tilde{b}_{l,k} &= y_{l,k} \cdot (-M) \\
    y_{l,k} &\in \{0,1\}, \quad y_{l,k} \in \Z
  \end{align}
  where $M$ is chosen such that $e^{-M} \approx 0$ (typically $M \in
  [10, 20]$), and $y_{l,k}$ is constrained to be in $\{0,1\}$ through
  some solver-dependent method. Classical Big-M or
  special-ordered-sets are options to constraint $y_{l,k}$.
\end{definition}

This encoding preserves binary semantics:

\begin{itemize}
    \item $\tilde{x}_{l,k} = 0 \Rightarrow e^{\tilde{x}_{l,k}} = 1$
    \item $\tilde{x}_{l,k} = -M \Rightarrow e^{\tilde{x}_{l,k}} \approx 0$
\end{itemize}

In our case, although we are interested in formulating our transition
function as a set of linear constraints to allow us to represent the
problem as a petri net and take advantage of the ability to discover
extra constraints that way, a mixed-integer linear program (MILP) is
acceptable and we simply define $y_{l,k}$ to be integer variables to
our solver. It is possible to further use the traditional Big-M
encoding instead to force $y_{l,k} \in \{0,1\}$ and pose the entire
problem as a strictly linear program.

\section{Petri Net Encoding and Analysis}

\subsection{Petri Net Structure}

The log-space MILP is encoded as a Petri Net $\mathcal{N} = (P, T, F, W, M_0)$:

\begin{itemize}
    \item \textbf{Places} $P$: State variables (log-probabilities, binary indicators)
    \item \textbf{Transitions} $T$: After GP condensation
      and subsequent log transform, our actions when a transition
      'fires' are to update a place for the next step with a weighted
      sum from the current step.
    \item \textbf{Flow relation} $F \subseteq (P \times T) \cup (T \times P)$: Arc structure
    \item \textbf{Weights} $W: F \to \R$: Arc weights from GP coefficients $\phi_{jk}$
    \item \textbf{Initial marking} $M_0$: Initial state
\end{itemize}

\section{Solution Recovery}

\subsection{Back-Transformation}

Given SMT solution $\vec{z}^*$, recover original variables:

\begin{enumerate}
    \item \textbf{Continuous variables:} $x_j = e^{z_j^*}$
    \item \textbf{Binary variables:}
    \begin{equation}
        b_{l,k} = \begin{cases}
            0 & \text{if } \tilde{x}_{l,k} < -M/2 \\
            1 & \text{otherwise}
        \end{cases}
    \end{equation}
    \item \textbf{Probabilities:} Map auxiliary variables to original
\end{enumerate}

\subsection{Validation}


\section{Theoretical Analysis}

\subsection{Correctness}

\begin{theorem}[Pipeline Correctness]
  Let $\mathcal{P}$ be the original MILP and $\mathcal{S}$ be the SMT
  problem generated by our pipeline. Then:
\begin{enumerate}
\item Every solution to $\mathcal{S}$ corresponds to a feasible
  solution of $\mathcal{P}$
\item The optimal solution of $\mathcal{S}$ is within $\epsilon$ of
  the optimal solution of $\mathcal{P}$, where $\epsilon = O(e^{-M})$
\end{enumerate}
\end{theorem}

\begin{proof}
The proof follows from:
\begin{enumerate}
    \item GP condensation is conservative (AM-GM inequality)
    \item Binary constraints are exactly preserved through Big-M encoding
    \item Petri Net analysis respects integer constraints (Z3 SAT checking)
    \item Solution recovery threshold ensures $e^{-M} \approx 0$
\end{enumerate}
\end{proof}

\subsection{Complexity}

\begin{proposition}[Complexity Analysis]
For a problem with $n$ locations and horizon $N$:
\begin{itemize}
    \item GP condensation: $O(n^2 N)$

    \item Petri Net analysis: $O(n^3 N^2)$

    \item SMT solving: Exponential worst-case, but reduced by discovered constraints.
\end{itemize}

How much the SMT solving is reduced in complexity/time cost depends on
the number and strength of discovered invariants.

\end{proposition}

\section{Experimental Evaluation}

\subsection{Benchmark Problems}


\subsection{Results}


\subsection{Analysis of Discovered Constraints}

% Further mysterious AI generated stuff.

% \begin{table}[h]
% \centering
% \caption{Structural Constraints Discovered}
% \begin{tabular}{|l|c|c|c|c|}
% \hline
% Problem & Mutexes & Invariants & Infeasible Trans. & Min Steps \\
% \hline
% Grid-5×5 & 300 & 12 & 45\% & 8 \\
% Corridor-20 & 190 & 8 & 62\% & 12 \\
% Museum-B & 512 & 23 & 38\% & 15 \\
% \hline
% \end{tabular}
% \end{table}

% The results demonstrate that Petri Net analysis discovers significant structural constraints, with 38-62\% of transitions marked as infeasible and hundreds of mutex relationships identified. These constraints reduce the search space dramatically, yielding order-of-magnitude speedups.

\section{Related Work}

\subsection{Geometric Programming in Planning}

While GP has been extensively studied for optimization
\cite{boyd2007tutorial}, its application to planning problems is
limited. Previous work has focused on:

\begin{itemize}
    \item Trajectory optimization with posynomial dynamics \cite{trajectory2018}
    \item Resource allocation in planning \cite{resource2019}
\end{itemize}

Our work differs by using GP purely for structural transformation, not optimization.

\subsection{Petri Net Analysis for Planning}

Petri Nets have been used for:
\begin{itemize}
    \item Concurrent action planning \cite{concurrent2020}
    \item Deadlock detection \cite{deadlock2021}
\end{itemize}

We extend these approaches by combining PN analysis with GP-derived structure.

\subsection{SMT-Based Planning}

Recent advances in SMT solving for planning include:
\begin{itemize}
    \item Lazy clause generation \cite{lazy2022}
    \item Symmetry breaking \cite{symmetry2023}
\end{itemize}

Our preprocessing complements these techniques.

\section{Discussion}

\subsection{When to Apply This Pipeline}

The GP-PN pipeline is most effective when:
\begin{enumerate}
    \item Problem has hidden structural constraints
    \item Planning horizon is moderate (10-50 steps)
    \item Bilinear terms arise from probability-location products
    \item Direct SMT solving struggles with combinatorics
\end{enumerate}

\subsection{Limitations}

\begin{itemize}
    \item Condensation quality depends on choice of $\hat{x}$
    \item Big-M encoding introduces numerical approximation
    \item Pipeline overhead may not amortize for small problems
    \item Requires posynomial-compatible formulation
\end{itemize}

\subsection{Extensions}

Future work could explore:
\begin{enumerate}
    \item Adaptive condensation point selection
    \item Iterative refinement with discovered solutions
    \item Application to other planning domains
    \item Integration with learning-based methods
\end{enumerate}

\section{Conclusions}

We have presented a novel pipeline that transforms probabilistic
planning problems into Petri Net structures via Geometric Programming
condensation. The key innovation is using GP not for optimization but
for creating analyzable weighted sum structures. By carefully managing
binary variables through Big-M encoding in log-space, we preserve
exact discrete semantics while enabling powerful structural analysis.

Our experimental results demonstrate order-of-magnitude speedups on
benchmark problems, with the Petri Net analysis discovering hundreds
of mutex relationships and marking large fractions of transitions as
infeasible. These structural constraints significantly reduce the
search space for the SMT solver.

The broader implication is that creative mathematical transformations
can unlock powerful analysis techniques even when problems don't
naturally fit the framework. By viewing GP condensation as a
preprocessing tool rather than an optimization method, we open new
avenues for combining disparate mathematical frameworks to tackle
challenging planning problems.

\section*{Acknowledgments}

We thank [redacted] for valuable discussions on Petri Net analysis and
[redacted] for insights on geometric programming theory.

\bibliographystyle{plain}
\bibliography{references}

% Note: You would need to create a references.bib file with entries like:
% @article{boyd2007tutorial,
%   title={A tutorial on geometric programming},
%   author={Boyd, Stephen and Kim, Seung-Jean and Vandenberghe, Lieven and Hassibi, Arash},
%   journal={Optimization and engineering},
%   volume={8},
%   pages={67--127},
%   year={2007},
%   publisher={Springer}
% }

\appendix

\section{Proof of Theorem 1}

\begin{proof}
We prove each part separately:

\textbf{Part 1:} Every solution to $\mathcal{S}$ corresponds to a feasible solution of $\mathcal{P}$.

Let $\vec{z}^*$ be a solution to $\mathcal{S}$. By construction:
\begin{enumerate}
    \item The GP condensation satisfies $\tilde{g}(x,\hat{x}) \leq g(x)$, so any point satisfying condensed constraints satisfies original posynomial constraints
    \item Binary enforcement constraints ensure $y_{l,i} \in \{0,1\}$
    \item Back-transformation with threshold recovers valid binary values
\end{enumerate}

Therefore, the recovered solution satisfies all constraints of $\mathcal{P}$.

\textbf{Part 2:} Optimality gap is $O(e^{-M})$.

The only approximation occurs in representing $b=0$ as $e^{-M}$. For sufficiently large $M$:
\begin{equation}
    |b - e^{-M}| < e^{-M}
\end{equation}

This error propagates linearly through the objective function, yielding total error $O(e^{-M})$.
\end{proof}

\section{Implementation Details}

\subsection{Parameter Selection}

\begin{table}[h]
\centering
\caption{Recommended Parameter Values}
\begin{tabular}{|l|l|l|}
\hline
Parameter & Range & Default \\
\hline
Big-M value & [10, 20] & 15 \\
Recovery threshold & $[-M/2, -M/3]$ & $-M/2$ \\
Condensation iterations & [1, 5] & 3 \\
SMT timeout & [60s, 3600s] & 600s \\
\hline
\end{tabular}
\end{table}

\subsection{Numerical Stability}

To ensure numerical stability:
\begin{enumerate}
    \item Check condition number of condensed problem
    \item Implement safeguards for near-zero denominators
    \item Use scaled arithmetic for extreme probabilities
    \item Monitor constraint violation after back-transformation
\end{enumerate}

\end{document}
